%   Copyright 2016 Emilio Rojas
%
%   Licensed under the Apache License, Version 2.0 (the "License");
%   you may not use this file except in compliance with the License.
%   You may obtain a copy of the License at
%
%       http://www.apache.org/licenses/LICENSE-2.0
%
%   Unless required by applicable law or agreed to in writing, software
%   distributed under the License is distributed on an "AS IS" BASIS,
%   WITHOUT WARRANTIES OR CONDITIONS OF ANY KIND, either express or implied.
%   See the License for the specific language governing permissions and
%   limitations under the License.

\bigskip

\textbf{Caso: $D_1$ on, $D_2$ off}

Se aplica el estado del diodo.
\begin{align*}
  I_{D1} &> 0
  &
  I_{D2} = V_{D1} &= 0
  &
  V_{D2} &\leq 0
\end{align*}

\begin{align}
  V_o &= V_i - 1k I_i - 1k I_{D1} - 6 \label{p3:onoff:vo1}
  \\
  V_o &= V_i - 3k I_i + 2k I_{D1} \label{p3:onoff:vo2}
  \\
  V_o &= V_{D2} - 1k I_i + 1k I_{D1} \label{p3:onoff:vo3}
  \\
  V_o &= 2 + 1k I_i \label{p3:onoff:vo4}
\end{align}

Se obtiene $I_{D1}$ e $V_{D2}$ de \ref{p3:onoff:vo1} y \ref{p3:onoff:vo3}.

\begin{align*}
  1k I_{D1} &= V_i - 1k I_i - V_o - 6
\end{align*}
\begin{align*}
  V_{D2} &= V_o - 1k I_i + 1k I_{D1}
  \\
  V_{D2} &= V_o - 1k I_i + V_i - 1k I_i - V_o - 6
  \\
  V_{D2} &= V_i - 2k I_i - 6
\end{align*}

Se obtiene $I_i$ de \ref{p3:onoff:vo2}

\begin{align*}
  V_o &= V_i - 3k I_i + 2k I_{D1}
  \\
  V_o &= V_i - 3k I_i + 2 \left(V_i - 1k I_i - V_o - 6\right)
  \\
  V_o &= V_i - 3k I_i + 2 V_i - 2k I_i - 2 V_o - 12
  \\
  3 V_o &= 3 V_i - 5k I_i - 12
  \\
  5k I_i &= 3 V_i - 3 V_o - 12
  \end{align*}

Se remplaza $I_i$ en $I_{D1}$ e $V_{D2}$

\begin{align*}
  1k I_{D1} &= V_i - 1k I_i - V_o - 6
  \\
  1k I_{D1} &= V_i - \left(\frac{3 V_i - 3 V_o - 12}{5}\right) - V_o - 6
  \\
  1k I_{D1} &= V_i - \frac{3}{5} V_i + \frac{3}{5} V_o + \frac{12}{5} - V_o - 6
  \\
  1k I_{D1} &= \frac{2}{5} V_i - \frac{2}{5} V_o - \frac{18}{5}
\end{align*}
\begin{align*}
  V_{D2} &= V_i - 2k I_i - 6
  \\
  V_{D2} &= V_i - 2 \left(\frac{3 V_i - 3 V_o - 12}{5}\right) - 6
  \\
  V_{D2} &= V_i - \frac{6}{5} V_i + \frac{6}{5} V_o + \frac{24}{5} - 6
  \\
  V_{D2} &= - \frac{1}{5} V_i + \frac{6}{5} V_o - \frac{6}{5}
\end{align*}

Se obtiene $V_o$ en función de $V_i$ con \ref{p3:onoff:vo4}

\begin{align*}
  V_o &= 2 + 1k I_i
  \\
  V_o &= 2 + \left(\frac{3 V_i - 3 V_o - 12}{5}\right)
  \\
  V_o &= 2 + \frac{3}{5} V_i - \frac{3}{5} V_o - \frac{12}{5}
  \\
  \frac{8}{5} V_o &= \frac{3}{5} V_i - \frac{2}{5}
  \\
  V_o &= \frac{3}{8} V_i - \frac{1}{4}
\end{align*}

Se aplican restricciones para obtener rango de funcionamiento.

\begin{align*}
  5k I_{D1} = 2 V_i - 2 V_o - 18 &> 0
  \\
  2 V_i - 2 V_o &> 18
  \\
  2 V_i - 2 \left(\frac{3}{8} V_i - \frac{1}{4}\right) &> 18
  \\
  2 V_i - \frac{3}{4} V_i + \frac{1}{2} &> 18
  \\
  \frac{5}{4} V_i &> \frac{35}{2}
  \\
  V_i &> 14
\end{align*}

\begin{align*}
  V_{D2} = - \frac{1}{5} V_i + \frac{6}{5} V_o - \frac{6}{5} &\leq 0
  \\
  - \frac{1}{5} V_i + \frac{6}{5} \left(\frac{3}{8} V_i - \frac{1}{4}\right) &\leq \frac{6}{5}
  \\
  - \frac{1}{5} V_i + \frac{9}{20} V_i - \frac{3}{10} &\leq \frac{6}{5}
  \\
  \frac{1}{4} V_i &\leq \frac{3}{2}
  \\
  V_i &\leq 6
\end{align*}


Caso no existe puesto que $\left]-\infty, 6\right] \quad \bigcup \quad \left]14, \infty\right[ = \emptyset$.
