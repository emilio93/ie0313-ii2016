%   Copyright 2016 Emilio Rojas
%
%   Licensed under the Apache License, Version 2.0 (the "License");
%   you may not use this file except in compliance with the License.
%   You may obtain a copy of the License at
%
%       http://www.apache.org/licenses/LICENSE-2.0
%
%   Unless required by applicable law or agreed to in writing, software
%   distributed under the License is distributed on an "AS IS" BASIS,
%   WITHOUT WARRANTIES OR CONDITIONS OF ANY KIND, either express or implied.
%   See the License for the specific language governing permissions and
%   limitations under the License.

\textbf{Ecuaciones para Vo}

Sea $I_i$ la corriente que pasa por la fuente sinusoidal $V_i$, se formulan las
ecuaciones por cuatro rutas hacia $V_o$:

% Todas las rutas comienzan en el negativo de V_o.

\begin{align}
  % Ruta por D1.
  V_o &= V_i - 1k I_i - V_{D1} - 1k I_{D1} - 6 \label{p3:inicio:vo1_pre}
  \\
  \nonumber \\
  % Ruta por las 3 resistencias
  V_o &= V_i - 1k I_i - 1k (I_i - I_{D1}) - 1k (I_i - I_{D1} - I_{D2}) \nonumber
  \\
  V_o &= V_i - 3k I_i + 2k I_{D1} + 1k I_{D2} \label{p3:inicio:vo2_pre}
  \\
  \nonumber \\
  % Ruta por D2
  V_o &= 1k I_{D2} + V_{D2} - 1k (I_i - I_{D1} - I_{D2}) \nonumber
  \\
  V_o &= 2k I_{D2} + V_{D2} - 1k I_i + 1k I_{D1} \label{p3:inicio:vo3_pre}
  \\
  \nonumber \\
  % Ruta por fuente de 2V.
  V_o &= 2 + 1k (I_i - I_{D2}) \nonumber
  \\
  V_o &= 2 + 1k I_i - 1k I_{D2} \label{p3:inicio:vo4_pre}
\end{align}

Se despeja $I_i$ de \ref{p3:inicio:vo2_pre} y \ref{p3:inicio:vo4_pre}:

\begin{align}
  V_i - 3k I_i + 2k I_{D1} + 1k I_{D2} &= 2 + 1k I_i - 1k I_{D2} \nonumber
  \\
  4k I_i &= V_i + 2k I_{D1} + 2k I_{D2} - 2 \label{p3:inicio:ii}
\end{align}

Se utiliza \ref{p3:inicio:ii} en \ref{p3:inicio:vo1_pre},
\ref{p3:inicio:vo2_pre}, \ref{p3:inicio:vo3_pre} y \ref{p3:inicio:vo4_pre} para
dejar estas fórmulas en términos $V_i$ y, los voltajes y corrientes
de los diodos:

\begin{align}
  % Ruta por D1.
  V_o &=
    V_i
    - \frac{1}{4} \left( V_i + 2k I_{D1} + 2k I_{D2} - 2 \right)
    - V_{D1}
    - 1k I_{D1}
    - 6 \nonumber
  \\
  V_o &=
    \frac{3}{4} V_i
    - V_{D1}
    - 1.5k I_{D1}
    - 500 I_{D2}
    - \frac{11}{2} \label{p3:inicio:vo1_pre2}
  \\
  \nonumber \\
  % Ruta por las 3 resistencias
  V_o &=
    V_i
    - \frac{3}{4} \left( V_i + 2k I_{D1} + 2k I_{D2} - 2 \right)
    + 2k I_{D1}
    + 1k I_{D2} \nonumber
  \\
  V_o &=
    \frac{1}{4} V_i
    + 500 I_{D1}
    - 500 I_{D2}
    + \frac{3}{2} \label{p3:inicio:vo2_pre2}
  \\
  \nonumber \\
  % Ruta por D2
  V_o &=
    2k I_{D2}
    + V_{D2}
    - \frac{1}{4} \left( V_i + 2k I_{D1} + 2k I_{D2} - 2 \right)
    + 1k I_{D1} \nonumber
  \\
  V_o &=
    1.5k I_{D2}
    + V_{D2}
    - \frac{1}{4} V_i
    + 500 I_{D1}
    + \frac{1}{2} \label{p3:inicio:vo3_pre2}
  \\
  \nonumber \\
  % Ruta por fuente de 2V.
  V_o &=
    2
    + \frac{1}{4} \left( V_i + 2k I_{D1} + 2k I_{D2} - 2 \right)
    - 1k I_{D2} \nonumber
  \\
  V_o &=
    \frac{3}{2}
    + \frac{1}{4} V_i
    - 500 I_{D2}
    + 500 I_{D1} \label{p3:inicio:vo4_pre2}
\end{align}

Se nota que \ref{p3:inicio:vo2_pre2} y \ref{p3:inicio:vo4_pre2} son la misma
ecuación, por lo que ahora se tiene solo 3 ecuaciones:

\begin{align}
  V_o &=
    \frac{3}{4} V_i
    - 1.5k I_{D1}
    - V_{D1}
    - 500 I_{D2}
    - \frac{11}{2} \label{p3:inicio:vo1}
  \\
  V_o &=
    \frac{1}{4} V_i
    + 500 I_{D1}
    - 500 I_{D2}
    + \frac{3}{2} \label{p3:inicio:vo2}
  \\
  V_o &=
    - \frac{1}{4} V_i
    + 500 I_{D1}
    + 1.5k I_{D2}
    + V_{D2}
    + \frac{1}{2} \label{p3:inicio:vo3}
\end{align}
