%   Copyright 2016 Emilio Rojas
%
%   Licensed under the Apache License, Version 2.0 (the "License");
%   you may not use this file except in compliance with the License.
%   You may obtain a copy of the License at
%
%       http://www.apache.org/licenses/LICENSE-2.0
%
%   Unless required by applicable law or agreed to in writing, software
%   distributed under the License is distributed on an "AS IS" BASIS,
%   WITHOUT WARRANTIES OR CONDITIONS OF ANY KIND, either express or implied.
%   See the License for the specific language governing permissions and
%   limitations under the License.

\bigskip

\textbf{Caso: $D_1$ off, $D_2$ on}

Se aplica el estado de los diodos:

\begin{align*}
  I_{D1} = V_{D2} &= 0
  &
  I_{D2} &> 0
  &
  V_{D1} &\leq 0
\end{align*}

\begin{align}
  V_o &=
    \frac{3}{4} V_i
    - V_{D1}
    - 500 I_{D2}
    - \frac{11}{2} \label{p3:offon:vo1}
  \\
  V_o &=
    \frac{1}{4} V_i
    - 500 I_{D2}
    + \frac{3}{2} \label{p3:offon:vo2}
  \\
  V_o &=
    - \frac{1}{4} V_i
    + 1.5k I_{D2}
    + \frac{1}{2} \label{p3:offon:vo3}
\end{align}

Se despeja $I_{D2}$ de \ref{p3:offon:vo2} y \ref{p3:offon:vo3}:

\begin{align*}
  \frac{1}{4} V_i   - 500 I_{D2}  + \frac{3}{2} &=
  - \frac{1}{4} V_i + 1.5k I_{D2} + \frac{1}{2}
  \\
  2k I_{D2} &= \frac{1}{2} V_i + 1
\end{align*}

Se despeja $V_{D1}$ de \ref{p3:offon:vo1} y \ref{p3:offon:vo2}:

\begin{align*}
  \frac{3}{4} V_i - V_{D1} - 500 I_{D2} - \frac{11}{2} &=
  \frac{1}{4} V_i          - 500 I_{D2} + \frac{3}{2}
  \\
  V_{D1} &= \frac{1}{2} V_i - 7
\end{align*}


Se obtiene $V_o(V_i)$ y $V_i(V_o)$ de \ref{p3:offon:vo3}:

\begin{align*}
  V_o &= - \frac{1}{4} V_i + \frac{3}{4} \left( \frac{1}{2} V_i + 1 \right) + \frac{1}{2}
  \\
  V_o &= \frac{1}{8} V_i + \frac{5}{4} = \frac{V_i + 10}{8} \Rightarrow V_i = 8 V_o - 10
\end{align*}

Se obtiene la restricción dada por $I_{D2}$:

\begin{align*}
  2k I_{D2} = \frac{1}{2} V_i + 1 > 0
  \Rightarrow
  V_i > -2
\end{align*}

Se obtiene la restricción dada por $V_{D1}$:

\begin{align*}
  V_{D1} = \frac{1}{2} V_i - 7 \leq 0
  \Rightarrow
  V_i \leq 14
\end{align*}

Se obtiene que:

\begin{equation} \label{p3:offon}
  V_o = \frac{V_i + 10}{8}
  \quad
  \quad
  \mathrm{si}
  \quad
  \quad
  -2 < V_i \leq 14
\end{equation}
