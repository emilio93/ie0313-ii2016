Encuentre los valores mínimo y máximo de $r_i$, si $V_i$ varía entre $80V$ y
$100V$.

\begin{figure}[H]
  \begin{center}
    \begin{circuitikz}

      \draw (0,0)
      to [sV, l=$V_i$] (0, 4)
      to [R, l=$r_i$] (3, 4)
      to [short] (9, 4)
      to [R, l_=$100\Omega$] (9, 0)
      to [short] (0, 0)
      ;

      \draw (3, 0)
      to [zzD*, l=$Z_2$] (3, 2)
      to [zzD*, l=$Z_1$] (3, 4)
      ;

      \draw (3, 2) to [short] (6, 2)
      ;

      \draw (6, 0)
      to [R, l=$430\Omega$] (6, 2)
      to [R, l=$200\Omega$] (6, 4)
      ;

    \end{circuitikz}
  \end{center}
\end{figure}

\begin{align*}
  Z_1 &: \mathrm{1N4740A} \\
  Z_2 &: \mathrm{1N4733A}
\end{align*}

\begin{center}
  \begin{tabular}{ | c | c c c c | c c c | c c c | c | c | }
    \hline
    \multirow{2}{*}{Device} &
    $V_Z$ & $Z_Z$ & \multirow{2}{*}{@} & $I_{ZT}$ &
    $Z_{ZK}$ & \multirow{2}{*}{@} & $I_{ZK}$ &
    $V_R$ & \multirow{2}{*}{@} & $I_R$ &
    $I_{\mathrm{SURGE}}$ &
    $I_{\mathrm{ZM}}$
    \\
    &
    ($V$) & ($\Omega$) & & ($mA$) &
    ($\Omega$) & & ($mA$) &
    ($V$) & & ($\mu A$) &
    ($mA$) &
    ($mA$)
    \\
    \hline
    1N4733A &
    5.1 & 7.0 & \multirow{2}{*}{.} & 49 &
    550 & \multirow{2}{*}{.} & 1.0 &
    1.0 & \multirow{2}{*}{.} & 10 &
    890 &
    178
    \\
    1N4740A &
    10 & 7.0 & & 25 &
    700 & & 0.25 &
    7.6 & & 10 &
    454 &
    91
    \\
    \hline
  \end{tabular}
\end{center}

\subsection{Solución}

\subsubsection{Ecuaciones e Información de los Diodos Zener}

\begin{align*}
  I_Z &= \frac{1}{Z_Z} V_Z + \left( I_{ZT} - \frac{V_{Z_T}}{Z_Z} \right) &
  V_Z &= Z_Z I_Z - Z_Z \left( I_{ZT} - \frac{V_{Z_T}}{Z_Z} \right)\\
\end{align*}
\begin{align*}
  Z_1 :& I_{Z1} = \frac{V_{Z1}}{7} - 1.40357
       & 0.00025 < I_{Z1} < 0.091\\
       & V_{Z1} = 7 \cdot I_{Z1} + 9.825
       & 9.827 < V_{Z1} < 10.462\\
  Z_2 :& I_{Z2} = \frac{V_{Z2}}{7} - 0.67957
       & 0.001 < I_{Z2} < 0.178 \\
       & V_{Z2} = 7 \cdot I_{Z2} + 4.757
       & 4.764 < V_{Z2} < 6.003\\
\end{align*}

\subsubsection{Ecuaciones del Circuito}

\begin{align}
  \frac{V_i - (V_{Z1}+V_{Z2})}{r_i} &= I_{100\Omega} + I_{200\Omega} + I_{Z1} \nonumber \\
  r_i &= \frac{V_i - (V_{Z1}+V_{Z2})}{I_{100\Omega} + I_{200\Omega} + I_{Z1}} \label{eq:r_i}
\end{align}

Definimos las corrientes por las resistencias de acuerdo a los voltajes de los
diodos:

\begin{align}
  I_{200\Omega} &= \frac{V_{Z1}}{200} \label{eq:i_r200} \\
  I_{430\Omega} &= \frac{V_{Z2}}{430} \label{eq:i_r430} \\
  I_{100\Omega} &= \frac{V_{Z1} + V_{Z2}}{100} \label{eq:i_r100}
\end{align}

Sustituimos \ref{eq:i_r200} y \ref{eq:i_r100} en \ref{eq:r_i} para obtener la
ecuación de $r_i$ en terminos de las variables de los diodos.

\begin{equation} \label{eq:r_ifinal}
  r_i = \frac{V_i - (V_{Z1}+V_{Z2})}{\frac{V_{Z1} + V_{Z2}}{100} + \frac{V_{Z1}}{200} + I_{Z1}}
\end{equation}

\subsubsection{Casos para $r_i$}

$r_{i\mathrm{MIN}}$
\begin{align*}
  V_{i_{\mathrm{MAX}}} &= 100 \\
  I_{Z1{\mathrm{MAX}}} &\Rightarrow I_{Z1} + I_{200\Omega} = 0.091 + \frac{10.462}{200} = 0.14331 \\
  0.14331 &= I_{Z2} + I_{430\Omega} = I_{Z2} + \frac{V_{Z2}}{430} = I_{Z2} + \frac{7 I_{Z2} + 4.757}{430} \\
  0.14331 &= \frac{430I_{Z2} + 7 I_{Z2} + 4.757}{430} \\
  I_{Z2} &= \frac{430 \cdot 0.14331 - 4.757}{430 + 7} = \frac{56.8663}{437} \\
  I_{Z2} &\approx 0.13013 \Rightarrow V_{Z2} \approx 5.668
\end{align*}


Se usan estos valores en \ref{eq:r_ifinal}:\\
\begin{equation*}
  r_{i\mathrm{MIN}} = \frac{100 - (10.462 + 5.668)}{\frac{10.462 + 5.668}{100} + \frac{10.462}{200} + 0.091}
  \approx 275.34\Omega
\end{equation*}

\bigskip

$r_{i\mathrm{MAX}}$\\
\begin{align*}
  V_{i_{\mathrm{MIN}}} &= 80 \\
  I_{Z2{\mathrm{MIN}}} &\Rightarrow I_{Z2} + I_{430\Omega} = 0.001 + \frac{4.764}{430} = 0.012079 \\
  0.012079 &= I_{Z1} + I_{200\Omega} = I_{Z1} + \frac{V_{Z1}}{200} = I_{Z1} + \frac{7 I_{Z1} + 9.827}{200} \\
  0.012079 &= \frac{200I_{Z1} + 7 I_{Z1} + 9.827}{200} \\
  I_{Z1} &= \frac{200 \cdot 0.012079 - 9.827}{200 + 7} = \frac{-4632}{207}
\end{align*}
Como el valor de $I_{Z1}$ está fuera de sus límites, no sire. Se debe usar $I_{Z1{\mathrm{MIN}}}$.
\begin{align*}
  I_{Z1{\mathrm{MIN}}} &\Rightarrow I_{Z1} + I_{200\Omega} = 0.00025 + \frac{9.827}{200} = 0.049385 \\
  0.049385 &= I_{Z2} + I_{430\Omega} = I_{Z2} + \frac{V_{Z2}}{430} = I_{Z2} + \frac{7 I_{Z2} + 4.757}{430} \\
  0.049385 &= \frac{430I_{Z2} + 7 I_{Z2} + 4.757}{430} \\
  I_{Z2} &= \frac{430 \cdot 0.049385 - 4.757}{430 + 7} \approx 0.037708 \Rightarrow V_{Z2} \approx 5.021
\end{align*}


Ahora $V_{Z2}$ si está dentro del rango aceptado. Se usan estos valores en \ref{eq:r_ifinal}:\\
\begin{equation*}
  r_{i\mathrm{MIN}} = \frac{80 - (9.827 + 5.021)}{\frac{9.827 + 5.021}{100} + \frac{9.827}{200} + 0.00025}
  \approx 329.28\Omega
\end{equation*}
