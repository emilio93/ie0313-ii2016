
\usubsection{Análisis DC}

\begin{figure}[H]
  \begin{center}
    \begin{circuitikz}

      \draw (6,4) [american voltages]
      to [short] (7.01, 4) to [open] (6, 4)
      to [R, l=$R_1 \equal 1M\Omega$] (6, 8)
      to [short] (13, 8)
      to [R, l=$R_{E1}$] (13, 4)
      to [short] (15, 4)
      to [short] (15, 3.5)
      to [open, v^=$V_o$] (15, 1.5)
      to [short] (15, 0)
      ;

      \draw (7.85, 8)
      to [short] (7.85, 4.75)
      to [open, l=$T_3$] (7.85, 3.25)
      to [R, l=$R_{E3}$] (7.85, 0)
      to [short] (13, 0)
      to [short] (13, 1.75)
      to [open, l_=$T_1$] (13, 3.25)
      to [short] (13, 4)
      ;

      \draw (11.5, 6)
      to [short] (10.5, 6)
      to [short] (10.5, 4)
      to [open, l=$T_2$] (10.5, 2.5)
      to [R, l=$R_{C2}$] (10.5, 0)
      to [open] (10.5, 2.5)
      to [short] (12.2, 2.5)
      ;

      \draw (7.85, 3.25)
      to [short] (9.7, 3.25)
      ;

      \draw (7.85, 8.5)
      to [short, l=$30V$] (7.85, 8.5)
      to [short, o-*] (7.85, 8)
      to [open] (7.85, 0)
      to [short, -o] (7.85, -0.5)
      to [short, l=$-30V$] (7.85, -0.5)
      ;

      % T1
      \draw (7.85, 4)
      node [npn]{}
      ;

      % T2
      \draw (10.5, 3.25)
      node [pnp]{}
      ;

      % T3
      \draw (13, 2.5)
      node [pnp]{}
      ;

      \draw (11.5, 6)
      node [ground]{}
      ;

      \draw(15, 0)
      node[ground]{}
      ;
    \end{circuitikz}
  \end{center}
\end{figure}

Como el emisor del transistor $T_2$ está en tierra(i.e. $V_{E2} = 0$), la base
del mismo transistor está en $V_{B2} = V_{E2} - V_{EB} = 0 - 0.7 = -0.7 V$, este
es el mismo voltaje al cual se encuentra el emisor del transistor $T_3$, así que
la base del mismo transistor está en $V_{B3} = V_{B2} + V_{EB} = -0.7 + 0.7 = 0$.

Ahora obtenemos el valor de la corriente de base del $T_3$:

\begin{equation*}
  I_{B3} = \frac{30V - V_{B3}}{1M\Omega} = \frac{30V}{1M\Omega} = 30\mu A
\end{equation*}

Según las ecuaciones del transistor obtenemos $I_{C3}$:

\begin{equation*}
  I_{C3} = \beta I_{B3} = 100 \cdot 30\mu A = 3mA
\end{equation*}

Y también podemos aproximar $I_{E3} \approx I_{C3}$ porque $\beta$ es grande.

Aplicamos LCK en el nodo que une $R_{E3}$ y la base de $T_2$.

\begin{equation*}
  I_{E3} = I_{R_{E3}} - I_{B2}\Rightarrow
  I_{B2} = I_{R_{E3}} - I_{E3} = \frac{0 - 0.7 - (-30)}{R_{E3}} - 3mA
\end{equation*}

\begin{equation*}
  I_{B2} = \frac{29.3V}{R_{E3}} - 3mA
\end{equation*}

Ahora tenemos una restricción para $R_{E3}$

\begin{equation*}
  I_{B2} = \frac{29.3V}{R_{E3}} - 3mA > 0 \Rightarrow
  \frac{29.3V}{R_{E3}} > 3mA
\end{equation*}

\begin{equation*}
  R_{E3} < \frac{29300}{3} \approx 9767\Omega
\end{equation*}

Según las ecuaciones del transistor obtenemos $I_{C2}$:

\begin{equation*}
  I_{C2} = \beta I_{B2} =
  100 \cdot \left( \frac{29.3V}{R_{E3}} - 3mA \right) =
  \frac{2930}{R_{E3}} - 0.3
\end{equation*}
\\

Aplicamos LCK ahora en el nodo que une $R_{C2}$ y la base de $T_1$:
\begin{equation*}
  I_{C2} = I_{RC2} - I_{B1} \Rightarrow I_{B1} = I_{RC2} - I_{C2}
\end{equation*}

\begin{equation*}
  I_{B1} = \frac{30 - I_{E1} R_{E1} - 0.7}{R_{C2}} - \frac{2930}{R_{E3}} + 0.3
\end{equation*}

Según las ecuaciones del transistor obtenemos $I_{C1}$:

\begin{equation*}
  I_{C1} = \beta I_{B1} =
  100 \cdot \left( \frac{30 - I_{E1} R_{E1} - 0.7}{R_{C2}} - \frac{2930}{R_{E3}} + 0.3 \right)
\end{equation*}

\begin{equation*}
  I_{C1} = \frac{2930 - 100 I_{E1} R_{E1}}{R_{C2}} - \frac{293k}{R_{E3}} + 30
\end{equation*}

Sabemos que $V_o = V_{EC1}$ y que $I_{E1} \approx I_{C1}$:

\begin{equation*}
  V_{EC1} =
  30 - I_{E1} R_{E1} - (-30) =
  60 - I_{C1} R_{E1} =
  60 - \frac{2930 R_{E1} - 100 I_{E1} R_{E1}^2}{R_{C2}} - \frac{293k R_{E1}}{R_{E3}} + 30 R_{E1}
\end{equation*}

Despejamos $I_{E1}$:

\begin{equation*}
  \frac{100 R_{E1}^2}{R_{C2}} I_{E1} = 60 - \frac{2930 R_{E1}}{R_{C2}} - \frac{293k R_{E1}}{R_{E3}} + 30 R_{E1} - V_{CE1}
\end{equation*}

\begin{equation*}
  I_{E1} =
  \frac{R_{C2}}{100 R_{E1}^2}
  \left(
    60
    - \frac{2930 R_{E1}}{R_{C2}}
    - \frac{293k R_{E1}}{R_{E3}}
    + 30 R_{E1} - V_{CE1}
  \right)
\end{equation*}

\begin{equation*}
  I_{E1} =
  \frac{60 R_{C2}}{100 R_{E1}^2}
  - \frac{2930}{100 R_{E1}}
  - \frac{293k R_{C2}}{100 R_{E1} R_{E3}}
  + \frac{30 R_{C2}}{100 R_{E1}}
  - V_{CE1} \frac{R_{C2}}{100 R_{E1}^2} \approx I_{C1}
\end{equation*}

\begin{equation*}
  \Rightarrow I_{CQ1} =
  \left(
    \frac{3 R_{C2}}{5 R_{E1}^2}
    + \frac{3 R_{C2} - 293}{10 R_{E1}}
    + \frac{2930 R_{C2}}{R_{E1} R_{E3}}
  \right)
  - V_{CEQ1} \frac{R_{C2}}{100 R_{E1}^2}
\end{equation*}

\begin{equation*}
  \Rightarrow I_{CQ1} =
  \left(
    \frac{3 R_{C2}}{5 R_{E1}^2}
    + \frac{3 R_{C2} - 293}{10 R_{E1}}
    + \frac{2930 R_{C2}}{R_{E1} R_{E3}}
  \right)
  - V_{CEQ1} \frac{R_{C2}}{100 R_{E1}^2}
\end{equation*}

Cuando $I_{C1} = 0$(caso de corte) se tiene inevitablemente que $I_{CE1} = 60V$.

Entonces para Máxima Excursión Simétrica se tiene que:

\begin{itemize}
  \item $I_{CQ1}(V_{CEQ1}=0) = 200mA$
  \item $V_{CEQ1}(I_{CQ1}=0) = 60V$
\end{itemize}

Se deduce la recta de carga DC $I_{CQ} = 0.2 - \frac{1}{300} V_{CEQ}$. Para que
el punto de operación Q esté a $100mA$ debe darse que $V_{CEQ} = 30V$, por lo
que $R_{E1} = 300\Omega$.

Obtenemos a partir de $I_CQ1$ y la recta de carga el siguiente sistema:

\begin{empheq}[left=\empheqlbrace]{align*}
  & 0.2 = \frac{R_{C2}}{150k} + \frac{3 R_{C2} - 293}{3k} + \frac{293 R_{C2}}{30 R_{E3}} \\
  & \frac{1}{300} = \frac{R_{C2}}{9M}
\end{empheq}

De la segunda ecuación obtenemos $R_{C2} = 30k\Omega$ y se reformula la primer ecuación.

\begin{equation*}
  0.2 =
  \frac{30k}{150k} + \frac{90k - 293}{3k} + \frac{8.79M}{30 R_{E3}} =
  \frac{90307}{3000} + \frac{293k}{R_{E3}}
\end{equation*}
