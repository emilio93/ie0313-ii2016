\usubsection{Análisis DC}

\begin{figure}[H]
  \begin{center}
    \begin{circuitikz}

      \draw (6,4) [american voltages]
      to [short] (7.01, 4) to [open] (6, 4)
      to [R, l=$R_1 \equal 1M\Omega$] (6, 8)
      to [short] (13, 8)
      to [R, l=$R_{E1}$] (13, 4)
      to [short] (15, 4)
      to [short] (15, 3.5)
      to [open, v^=$V_o$] (15, 1.5)
      to [short] (15, 0)
      ;

      \draw (7.85, 8)
      to [short] (7.85, 4.75)
      to [open, l=$T_3$] (7.85, 3.25)
      to [R, l=$R_{E3}$] (7.85, 0)
      to [short] (13, 0)
      to [short] (13, 1.75)
      to [open, l_=$T_1$] (13, 3.25)
      to [short] (13, 4)
      ;

      \draw (11.5, 6)
      to [short] (10.5, 6)
      to [short] (10.5, 4)
      to [open, l=$T_2$] (10.5, 2.5)
      to [R, l=$R_{C2}$] (10.5, 0)
      to [open] (10.5, 2.5)
      to [short] (12.2, 2.5)
      ;

      \draw (7.85, 3.25)
      to [short] (9.7, 3.25)
      ;

      \draw (7.85, 8.5)
      to [short, l=$30V$] (7.85, 8.5)
      to [short, o-*] (7.85, 8)
      to [open] (7.85, 0)
      to [short, -o] (7.85, -0.5)
      to [short, l=$-30V$] (7.85, -0.5)
      ;

      % T1
      \draw (7.85, 4)
      node [npn]{}
      ;

      % T2
      \draw (10.5, 3.25)
      node [pnp]{}
      ;

      % T3
      \draw (13, 2.5)
      node [pnp]{}
      ;

      \draw (11.5, 6)
      node [ground]{}
      ;

      \draw(15, 0)
      node[ground]{}
      ;
    \end{circuitikz}
  \end{center}
\end{figure}

Se comienza averigüando la corriente $I_{E3}$:

\begin{equation*}
  I_{E3} = \beta I_{1M\Omega} = 100 \cdot \frac{30 - 0.7 + 0.7}{1M} = 3mA
\end{equation*}

Se aplica LCK en el nodo de $I_{E3}$, $I_{B2}$ y $R_{E3}$, para obtener la
ecuación de $R_{E3}$:

\begin{equation*}
  I_{E3} + I_{B2} = I_{R_{E3}} \Rightarrow 3mA + I_{B2} = \frac{-0.7 - (-30)}{R_{E3}}
\end{equation*}
\begin{equation*}
  R_{E3} = \frac{29.3}{3mA + I_{B2}}
\end{equation*}

Se diseña para que la corriente $I_{B2}$ sea la igual a
$I_{B1} = \frac{I_{C1}}{\beta} \approx \frac{I_{E1}}{100} = 1mA$. Se
obtiene que $R_{E3} = \frac{29.3}{4mA} = 7325\Omega$.

Se sabe que $I_{C2} = \beta I_{B2} = 100mA$, entonces se aplica LCK en el nodo
de $I_{C2}$, $I_{B1}$ y $I_{R_{C2}}$, para obtener la ecuación de $R_{C2}$:

\begin{equation*}
  I_{C2} + I_{B1} = I_{R_{C2}} \Rightarrow 100mA + 1mA = \frac{60 - I_{E1} R_{E1} - 0.7}{R_{C2}}
\end{equation*}
\begin{equation*}
  R_{C2} = \frac{59.3 - 100mA \cdot R_{E1}}{101mA}
\end{equation*}

$I_{E1}$ se ha utilizado en $100mA$ porque se quiere la Máxima Excursión
Simétrica que se da al tener esta corriente.

Para cuando $I_{C1} = 0$, se tendrá que
$V_{CE1} = 60V$(LTK), y cuando $V_{CE1} = 0$, $I_{C1} = 2 I_{E1} = 200mA$. Con
esta información se obtiene la recta de carga:
\begin{equation*}
  I_{C1}(V_{CE1}) = 0.2 - \frac{1}{300} V_{CE1}
\end{equation*}
\begin{equation*}
  V_{CE1}(I_{C1}) = 60 - 300 I_{C1}
\end{equation*}

Para lograr $V_{CE1}(I_{C1} = 100mA) = 30V$, la resistencia $R_{E1}$ debe ser de
$300\Omega$(LTK).

Finalmente se utiliza el valor encontrado de $R_{E1}$ para obtener $R_{C2}$:
\begin{equation*}
  R_{C2} = \frac{59.3 - 100mA \cdot R_{E1}}{101mA} = \frac{29.3}{0.101} \approx 290\Omega
\end{equation*}

El valor de $r_i$, así como el de $C$, no afectan para obtener la Máxima
Excursión Simétrica según lo solicitado.

\usubsection{Resuesta}

\begin{align*}
  R_{E1} &= 300 \Omega \\
  R_{C2} &= 290 \Omega \\
  R_{E3} &= 7325 \Omega
\end{align*}
